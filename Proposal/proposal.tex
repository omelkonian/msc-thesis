\documentclass[acmsmall,nonacm=true,screen=true]{acmart}\settopmatter{printfolios=false,printccs=false,printacmref=false}

%%%%%%%%%%%%%%%%%%%%%%%%%%%%%%%%%%%%%%%%%%
%% Bibliography style
%%%%%%%%%%%%%%%%%%%%%%%%%%%%%%%%%%%%%%%%%%
\bibliographystyle{ACM-Reference-Format}
\citestyle{acmauthoryear}

%%%%%%%%%%%%%%%%%%%%%%%%%%%%%%%%%%%%%%%%%%
%% Macros
%%%%%%%%%%%%%%%%%%%%%%%%%%%%%%%%%%%%%%%%%%
\newcommand\todo[1]{\textcolor{red}{TODO: #1}}
\newcommand\todocite[0]{}

%%%%%%%%%%%%%%%%%%%%%%%%%%%%%%%%%%%%%%%%%%
%% Packages
%%%%%%%%%%%%%%%%%%%%%%%%%%%%%%%%%%%%%%%%%%

% URLs
\newcommand\site[1]{\footnote{\url{#1}}}

% Quotes
\usepackage{csquotes}

% Font
%\usepackage{fontspec}
%\setmonofont{FiraCode-Light}

% Graphics
\graphicspath{ {images/} }
\usepackage{colortbl}

% Agda formatting
\usepackage{agda}
\newcommand\nat{
\begin{code}%
\>[0]\AgdaKeyword{data}\AgdaSpace{}%
\AgdaDatatype{ℕ}\AgdaSpace{}%
\AgdaSymbol{:}\AgdaSpace{}%
\AgdaPrimitiveType{Set}\AgdaSpace{}%
\AgdaKeyword{where}\<%
\\
\>[0][@{}l@{\AgdaIndent{0}}]%
\>[2]\AgdaInductiveConstructor{zero}%
\>[8]\AgdaSymbol{:}\AgdaSpace{}%
\AgdaDatatype{ℕ}\<%
\\
%
\>[2]\AgdaInductiveConstructor{suc}%
\>[8]\AgdaSymbol{:}\AgdaSpace{}%
\AgdaDatatype{ℕ}\AgdaSpace{}%
\AgdaSymbol{→}\AgdaSpace{}%
\AgdaDatatype{ℕ}\<%
\\
%
\\[\AgdaEmptyExtraSkip]%
\>[0]\AgdaOperator{\AgdaFunction{\AgdaUnderscore{}+\AgdaUnderscore{}}}\AgdaSpace{}%
\AgdaSymbol{:}\AgdaSpace{}%
\AgdaDatatype{ℕ}\AgdaSpace{}%
\AgdaSymbol{→}\AgdaSpace{}%
\AgdaDatatype{ℕ}\AgdaSpace{}%
\AgdaSymbol{→}\AgdaSpace{}%
\AgdaDatatype{ℕ}\<%
\\
\>[0]\AgdaInductiveConstructor{zero}%
\>[7]\AgdaOperator{\AgdaFunction{+}}\AgdaSpace{}%
\AgdaBound{n}\AgdaSpace{}%
\AgdaSymbol{=}\AgdaSpace{}%
\AgdaBound{n}\<%
\\
\>[0]\AgdaInductiveConstructor{suc}\AgdaSpace{}%
\AgdaBound{m}%
\>[7]\AgdaOperator{\AgdaFunction{+}}\AgdaSpace{}%
\AgdaBound{n}\AgdaSpace{}%
\AgdaSymbol{=}\AgdaSpace{}%
\AgdaInductiveConstructor{suc}\AgdaSpace{}%
\AgdaSymbol{(}\AgdaBound{m}\AgdaSpace{}%
\AgdaOperator{\AgdaFunction{+}}\AgdaSpace{}%
\AgdaBound{n}\AgdaSymbol{)}\<%
\end{code}
}



\begin{document}

%%%%%%%%%%%%%%%%%%%%%%%%%%%%%%%%%%%%%%%%%%
%% Title
%%%%%%%%%%%%%%%%%%%%%%%%%%%%%%%%%%%%%%%%%%
\title{Formal investigation of the expressiveness of the Extended UTxO model}
\subtitle{Laying the foundations for the formal verification of smart contracts}

%%%%%%%%%%%%%%%%%%%%%%%%%%%%%%%%%%%%%%%%%%
%% Authors
%%%%%%%%%%%%%%%%%%%%%%%%%%%%%%%%%%%%%%%%%%
\author{Orestis Melkonian}
\orcid{0000-0003-2182-2698}
\affiliation{
  \position{MSc Student}
  \department{Information and Computing Sciences}
  \institution{Utrecht University}
  \city{Utrecht}
  \country{The Netherlands}
}
\email{melkon.or@gmail.com}

%%%%%%%%%%%%%%%%%%%%%%%%%%%%%%%%%%%%%%%%%%
%% Abstract
%%%%%%%%%%%%%%%%%%%%%%%%%%%%%%%%%%%%%%%%%%
\begin{abstract}
This report serves as the proposal of my MSc thesis, supervised by Wouter Swierstra from
Utrecht University and Manuel Chakravarty from IOHK.
\end{abstract}

\maketitle

%%%%%%%%%%%%%%%%%%%%%%%%%%%%%%%%%%%%%%%%%%
\section{Introduction}
\label{sec:intro}
%%%%%%%%%%%%%%%%%%%%%%%%%%%%%%%%%%%%%%%%%%

% Distributed ledger technology creates issues
Although blockchain technology has opened a whole array of interesting new applications
(e.g. secure multi-party computation\cite{mpc}, fair protocol design fair\cite{fair}, zero-knowledge proof systems\cite{zeroproof}), 
reasoning about the behaviour of such systems is an exceptionally hard task. This is partly due to their concurrent nature, but
also the fiscal nature of the majority of the applications, which require a much higher degree of rigorousness compared to
conventional IT applications.

% Smart contracts create issues
The advent of smart contracts (programs that run on the blockchain itself) gave
rise to another source of vulnerabilities. Since these (possibly Turing-complete)
programs often deal with transactions of significant funds,
it is of utmost importance that one can reason and ideally provide formal proofs
about their behaviour in a concurrent/distributed setting.

% Aim of thesis
\paragraph{Research Question}
The aim of this thesis is to provide a mechanized formal model of an abstract distributed ledger equipped with
smart contracts, in which one can begin to formally investigate the expressiveness of the extended UTxo model.
Moreover, we hope to lay down firm grounds, onto which one can further conduct a formal comparison with account-based
models used in Ethereum. Put concisely, the research question posed is:
\begin{displayquote}
	\textit{How much expressiveness do we gain by extending the UTxO model?} \\
	\textit{Is it as expressive as the account-based model used in Ethereum?}
\end{displayquote}

\paragraph{Overview}
% Background
Section~\ref{sec:background} reviews some basic definitions related to blockchain
technology and introduces important literature, which will be the main subject of study
throughout the development of our reasoning framework.
% Methodology
Section~\ref{sec:methodology} describes the technology we will use to formally reason
about the problem at hand and some key design decisions we set upfront.
% Results
Section~\ref{sec:results} presents the progress made thus far in terms of (mechanized) formal verification,
as well as problems we have encountered and also expect along the way.
% Planning
Section~\ref{sec:plan} discusses next steps for the remainder of the thesis, as well as a rough estimate
on when these milestones will be completed.

%%%%%%%%%%%%%%%%%%%%%%%%%%%%%%%%%%%%%%%%%%
\section{Background}
\label{sec:background}
%%%%%%%%%%%%%%%%%%%%%%%%%%%%%%%%%%%%%%%%%%

\subsection{Distributed Ledger Technology: Blockchain}

\subsection{Smart Contracts}

\subsection{UTxO-based: Bitcoin}
\paragraph{\textsc{SCRIPT}}
...

\paragraph{The BitML Calculus}
Bad documentation ... scarcity of formal models ... bitML ...

\paragraph{Extended UTxO}
... 

\subsection{Account-based: Ethereum}
\paragraph{Solidity}

%%%%%%%%%%%%%%%%%%%%%%%%%%%%%%%%%%%%%%%%%%
\section{Methodology}
\label{sec:methodology}
%%%%%%%%%%%%%%%%%%%%%%%%%%%%%%%%%%%%%%%%%%

\subsection{Scope}
... no cryptographic/implementation details ...
... focus on the big picture ...

\subsection{Proof Mechanization}
Through mechanization ...
... vs informal mathematics ...

It is exactly this side effect, that will allow us
to discover edge cases and increase the confidence
of the model under investigation.

As our proof development vehicle, we choose Agda\todocite ...

\subsection{Agda}

... stay on a highly abstract level ... postulate cryptographic operations etc ...

\paragraph{Limitation}
... proof automation ala Coq ...

\subsection{The IOHK approach}

... rigorous methodology ...
... industry co-existing with academia ...
... formal verification (Agda/Coq) -> prototype/reference implementation (Haskell) -> production codebase (Haskell) ...

\subsection{Functional first}

... UTxO vs Account ...
... functional vs imperative ...
... dataflow vs ?? ...

%%%%%%%%%%%%%%%%%%%%%%%%%%%%%%%%%%%%%%%%%%
\section{Preliminary Results}
\label{sec:results}
%%%%%%%%%%%%%%%%%%%%%%%%%%%%%%%%%%%%%%%%%%

\subsection{Formal Model I: Extended UTxO}
\subsubsection{Inherently-typed validity of transactions}
\subsubsection{Scripts via Denotational Semantics}
\subsubsection{Address space as module parameter}
\subsubsection{Weakening Lemma}
\subsubsection{Example}

\subsection{Formal Model II: BitML Calculus}
\subsubsection{Contracts in BitML}
\subsubsection{Small-step Semantics}
... mention paper bug in [C-Control] ...
\subsubsection{Configurations modulo permutation}
\subsubsection{Example}

\subsection{Expected Problems}
... up to permutation -> quotient types -> homotopy type theory ...

%%%%%%%%%%%%%%%%%%%%%%%%%%%%%%%%%%%%%%%%%%
\section{Planning}
\label{sec:plan}
%%%%%%%%%%%%%%%%%%%%%%%%%%%%%%%%%%%%%%%%%%

\subsection{Extended UTxO}
... multi-currency ... 

\subsection{BitML Calculus}
... symbolic runs ... computational runs ... coherence

\subsection{UTxO-BitML Integration}

\subsection{Featherweight Solidity}

\subsection{Formal Comparison}
... lots of examples ...

\subsection{Proof Automation}

\subsection{Timetable}

\begin{figure*}
  \centering
  \newcommand{\months}[1]{\multicolumn{#1}{c}{\cellcolor{teal}} \\}
  \begin{tabular}{lcccccccc}
    \hline
    & 2018 & \multicolumn{7}{c}{2019} \\
    \hline
    & Dec & Jan & Feb & Mar & Apr & May & Jun & Jul \\
    \hline

    \textbf{UTxO} \\
    Basic abstract model                &      \months{2}
    Weakening lemma                     &&     \months{1}
    Extend with data scripts            &&&    \months{1}
    Multi-currency                      &&&&   \months{1}

    \textbf{BitML} \\
    Small-step Semantics                &      \months{3}
    Symbolic Model                      &&&&   \months{1}
    Computational Model                 &&&&   \months{1}
    Coherence                           &&&&&  \months{2}
    Computational Soundness             &&&&&  \months{2}

    \textbf{UTxO-BitML Integration} \\
    Compiling BitML contracts to UTxO   &&&&   \months{2}
    
    \textbf{Solidity} \\
    Featherweight Model                 &&&&   \months{1}
    Comparison framework                &&&&&  \months{2}
    
    \textbf{Evaluation} \\
    Example contracts                   &&&&&  \months{3}
    
    \textbf{Writeup} \\
    Write final thesis                  &&&&&& \months{3}
    
  \end{tabular}
  \caption{My workplan.}
  \label{fig:workplan}
\end{figure*}

%%%%%%%%%%%%%%%%%%%%%%%%%%%%%%%%%%%%%%%%%%
%% Bibliography
%%%%%%%%%%%%%%%%%%%%%%%%%%%%%%%%%%%%%%%%%%
\nocite{*} % T0D0 remove
\bibliography{sources}

\end{document}
