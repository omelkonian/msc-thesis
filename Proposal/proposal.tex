\documentclass[acmsmall,nonacm=true,screen=true]{acmart}
\settopmatter{printfolios=false,printccs=false,printacmref=false}

%%%%%%%%%%%%%%%%%%%%%%%%%%%%%%%%%%%%%%%%%%
%% Bibliography style
%%%%%%%%%%%%%%%%%%%%%%%%%%%%%%%%%%%%%%%%%%
\bibliographystyle{ACM-Reference-Format}
\citestyle{acmauthoryear}

%%%%%%%%%%%%%%%%%%%%%%%%%%%%%%%%%%%%%%%%%%
%% Macros
%%%%%%%%%%%%%%%%%%%%%%%%%%%%%%%%%%%%%%%%%%
\newcommand\todo[1]{\textcolor{red}{TODO: #1}}
\newcommand\todocite[0]{}
\def\bitcoin{%
  \leavevmode
  \vtop{\offinterlineskip %\bfseries
    \setbox0=\hbox{B}%
    \setbox2=\hbox to\wd0{\hfil\hskip-.03em
    \vrule height .3ex width .15ex\hskip .08em
    \vrule height .3ex width .15ex\hfil}
    \vbox{\copy2\box0}\box2}}


%%%%%%%%%%%%%%%%%%%%%%%%%%%%%%%%%%%%%%%%%%
%% Packages
%%%%%%%%%%%%%%%%%%%%%%%%%%%%%%%%%%%%%%%%%%

% URLs
\newcommand\site[1]{\footnote{\url{#1}}}

% Quotes
\usepackage{csquotes}

% Font
%\usepackage{fontspec}
%\setmonofont{FiraCode-Light}

% Graphics
\graphicspath{ {images/} }
\usepackage{colortbl}

% Agda formatting
\usepackage{agda}
\newcommand\nat{
\begin{code}%
\>[0]\AgdaKeyword{data}\AgdaSpace{}%
\AgdaDatatype{ℕ}\AgdaSpace{}%
\AgdaSymbol{:}\AgdaSpace{}%
\AgdaPrimitiveType{Set}\AgdaSpace{}%
\AgdaKeyword{where}\<%
\\
\>[0][@{}l@{\AgdaIndent{0}}]%
\>[2]\AgdaInductiveConstructor{zero}%
\>[8]\AgdaSymbol{:}\AgdaSpace{}%
\AgdaDatatype{ℕ}\<%
\\
%
\>[2]\AgdaInductiveConstructor{suc}%
\>[8]\AgdaSymbol{:}\AgdaSpace{}%
\AgdaDatatype{ℕ}\AgdaSpace{}%
\AgdaSymbol{→}\AgdaSpace{}%
\AgdaDatatype{ℕ}\<%
\\
%
\\[\AgdaEmptyExtraSkip]%
\>[0]\AgdaOperator{\AgdaFunction{\AgdaUnderscore{}+\AgdaUnderscore{}}}\AgdaSpace{}%
\AgdaSymbol{:}\AgdaSpace{}%
\AgdaDatatype{ℕ}\AgdaSpace{}%
\AgdaSymbol{→}\AgdaSpace{}%
\AgdaDatatype{ℕ}\AgdaSpace{}%
\AgdaSymbol{→}\AgdaSpace{}%
\AgdaDatatype{ℕ}\<%
\\
\>[0]\AgdaInductiveConstructor{zero}%
\>[7]\AgdaOperator{\AgdaFunction{+}}\AgdaSpace{}%
\AgdaBound{n}\AgdaSpace{}%
\AgdaSymbol{=}\AgdaSpace{}%
\AgdaBound{n}\<%
\\
\>[0]\AgdaInductiveConstructor{suc}\AgdaSpace{}%
\AgdaBound{m}%
\>[7]\AgdaOperator{\AgdaFunction{+}}\AgdaSpace{}%
\AgdaBound{n}\AgdaSpace{}%
\AgdaSymbol{=}\AgdaSpace{}%
\AgdaInductiveConstructor{suc}\AgdaSpace{}%
\AgdaSymbol{(}\AgdaBound{m}\AgdaSpace{}%
\AgdaOperator{\AgdaFunction{+}}\AgdaSpace{}%
\AgdaBound{n}\AgdaSymbol{)}\<%
\end{code}
}



\begin{document}

%%%%%%%%%%%%%%%%%%%%%%%%%%%%%%%%%%%%%%%%%%
%% Title
%%%%%%%%%%%%%%%%%%%%%%%%%%%%%%%%%%%%%%%%%%
\title{Formal investigation of the Extended UTxO model}
\subtitle{Laying the foundations for the formal verification of smart contracts}

%%%%%%%%%%%%%%%%%%%%%%%%%%%%%%%%%%%%%%%%%%
%% Authors
%%%%%%%%%%%%%%%%%%%%%%%%%%%%%%%%%%%%%%%%%%
\author{Orestis Melkonian}
\orcid{0000-0003-2182-2698}
\affiliation{
  \position{MSc Student}
  \department{Information and Computing Sciences}
  \institution{Utrecht University}
  \city{Utrecht}
  \country{The Netherlands}
}
\email{melkon.or@gmail.com}

%%%%%%%%%%%%%%%%%%%%%%%%%%%%%%%%%%%%%%%%%%
%% Abstract
%%%%%%%%%%%%%%%%%%%%%%%%%%%%%%%%%%%%%%%%%%
\begin{abstract}
This report serves as the proposal of my MSc thesis, supervised by Wouter Swierstra from
Utrecht University and Manuel Chakravarty from IOHK.
\end{abstract}

\maketitle

%%%%%%%%%%%%%%%%%%%%%%%%%%%%%%%%%%%%%%%%%%
\section{Introduction}
\label{sec:intro}
%%%%%%%%%%%%%%%%%%%%%%%%%%%%%%%%%%%%%%%%%%

% Distributed ledger technology creates issues
Although blockchain technology has opened a whole array of interesting new applications
(e.g. secure multi-party computation\cite{mpc}, fair protocol design fair\cite{fair}, zero-knowledge proof systems\cite{zeroproof}), 
reasoning about the behaviour of such systems is an exceptionally hard task. This is partly due to their concurrent nature, but
also the fiscal nature of the majority of the applications, which require a much higher degree of rigorousness compared to
conventional IT applications.

% Smart contracts create issues
The advent of smart contracts (programs that run on the blockchain itself) gave
rise to another source of vulnerabilities.
One primary example of such a vulnerability caused by the use of smart contracts is the
DAO attack\site{https://en.wikipedia.org/wiki/The_DAO_(organization)},
where a security flaw on the model of Ethereum's scripting language led to the exploitation of a venture capital fund
worth 150 million dollars at the time.
The solution was to create a hard fork of the Ethereum blockchain, clearly going against the decentralized spirit
of cryptocurrencies.
Since these (possibly Turing-complete) programs often deal with transactions of significant funds,
it is of utmost importance that one can reason and ideally provide formal proofs about their behaviour
in a concurrent/distributed setting.

% Aim of thesis
\paragraph{Research Question}
The aim of this thesis is to provide a mechanized formal model of an abstract distributed ledger equipped with
smart contracts, in which one can begin to formally investigate the expressiveness of the extended UTxo model.
Moreover, we hope to lay down firm grounds, onto which one can further conduct a formal comparison with account-based
models used in Ethereum. Put concisely, the research question posed is:
\begin{displayquote}
	\textit{How much expressiveness do we gain by extending the UTxO model?} \\
	\textit{Is it as expressive as the account-based model used in Ethereum?}
\end{displayquote}

\paragraph{Overview}
% Background
Section~\ref{sec:background} reviews some basic definitions related to blockchain
technology and introduces important literature, which will be the main subject of study
throughout the development of our reasoning framework.
% Methodology
Section~\ref{sec:methodology} describes the technology we will use to formally reason
about the problem at hand and some key design decisions we set upfront.
% Results
Section~\ref{sec:results} presents the progress made thus far in terms of (mechanized) formal verification,
as well as problems we have encountered and also expect along the way.
% Planning
Section~\ref{sec:plan} discusses next steps for the remainder of the thesis, as well as a rough estimate
on when these milestones will be completed.

%%%%%%%%%%%%%%%%%%%%%%%%%%%%%%%%%%%%%%%%%%
\section{Background}
\label{sec:background}
%%%%%%%%%%%%%%%%%%%%%%%%%%%%%%%%%%%%%%%%%%

% ?? Add cryptography section for:
%   1. hashes
%   2. private-pub key pairs for encryption/authentication

\subsection{Distributed Ledger Technology: Blockchain} \label{subsec:dlt}
Cryptocurrencies rely on distributed ledgers, where there is no central authority managing the accounts
and keeping track of the history of transactions.

One particular instance of distributed ledgers are blockchain systems, where (unrelated) transaction are
bundled together in blocks, which are linearly connected with hashes and distributed to all participants/peers.
The blockchain system, along with a consensus protocol deciding on which competing fork of the chain is to be included,
maintains an immutable distributed ledger (i.e. the history of transactions).

Validity of the transactions is tightly coupled with a consensus protocol, which makes sure
peers in the network only validate well-behaved/truthful transactions and are, moreover,
properly incentivized to do so.

The absence of a single central authority that has control over all assets of the participants allows
for shared control of the evolution of data (in this case transactions)
and generally leads to more robust and fair management of assets.

While cryptocurrencies are the major application of blockchain systems, one could easily
use them for any kind of valuable assets, or even as general distributed databases.

\subsection{Smart Contracts} \label{subsec:smartcontracts}
Most blockchain systems come equipped with a scripting language, where one can write
\textit{smart contracts} that dictate how a transaction operates. A smart contract
could, for instance, pose restrictions on who can redeem the output funds of a transaction.

One could view smart contracts as a replacement of legal frameworks, providing the means
to conduct contractual relationships completely algorithmically.

While previous work on writing financial contracts~\cite{spj} suggests it
is fairly straightforward to write such programs embedded in a general-purpose
language (in this case Haskell) and to reason about them with \textit{equational reasoning},
it is restricted in the centralized setting and, therefore, does not suffice for our needs.

Things become much more complicated when we move to the distributed/blockchain setting, as
can be evidenced by the current attempts being made to overcome this~\cite{setzer,short,scilla}.
Hence, there is a growing need for methods and tools that will
enable tractable and precise reasoning about such systems.

Numerous scripting languages have appeared recently~\cite{scriptlangs}, spanning a wide
spectrum of expressiveness and complexity. While language design can impose restrictions
on what can a language express, most of these restriction are inherited from
the accounting model that the underlying system adhere to.

In the next section, we will discuss the two main forms of accounting models:
\begin{enumerate}
\item \textbf{UTxO-based}: stateless models based on \textit{unspent transaction outputs}
\item \textbf{Account-based}: stateful models that explicitly model interaction between \textit{user accounts}
\end{enumerate}

\subsection{UTxO-based: Bitcoin} \label{subsec:bitcoin}
The primary example of a UTxO-based blockchain is Bitcoin~\cite{bitcoin}.
Its blockchain is a linear sequence of \textit{blocks} of transactions,
starting from the initial \textit{genesis} block.
Essentially, the blockchain acts as public log of all transactions that have taken place, where
each transaction refers to outputs of previous transactions,
except for the initial \textit{coinbase} transaction of each block.
Coinbase transactions have no inputs, create new currency and reward the miner of that block with a fixed amount.
Bitcoin also provides a cryptographic protocol to make sure no adversary can tamper with the transactional history,
e.g. by making the creation of new blocks computationally hard and invalidating the "truthful" chain statistically impossible.

A crucial aspect of Bitcoin's design is that there are no explicit addresses included in the transactions.
Rather, transaction outputs are actually program scripts, which allow someone to claim the funds by giving the proper inputs.
Thus, although there are no explicit user accounts in transactions, the effective available funds of a user
are all the \textit{unspent transaction outputs} (UTxO) that he can claim (e.g. by providing a digital signature).

\subsubsection{\textsc{Script}}
In order to state such scripts in the outputs of a transaction, Bitcoin provides a low-level, Forth-like,
stack-based scripting language, called \textsc{Script}.
\textsc{Script} is intentionally not Turing-complete (e.g. it does not provide looping structures),
in order to have more predictable behaviour.
Moreover, only a very restricted set of ``template" programs are considered standard, i.e.
allowed to be relayed from node to node.

\newcommand\ttt{\texttt}
\newcommand\stack[1]{\text{\ttt{#1}}}
\newcommand\Semantics[1]{\llbracket \stack{#1} \rrbracket}

\paragraph{P2PKH}
The most frequent example of a 'standard' program in \textsc{Script} is the
\textit{pay-to-pubkey-hash} (P2PKH) type of scripts. Given a hash of public key \texttt{<pub\#>},
a P2PKH output carries the following script:
\[
  \stack{OP\_DUP OP\_HASH <pub\#> OP\_EQ OP\_CHECKSIG}
\]
where \ttt{OP\_DUP} duplicates the top element of the stack, \ttt{OP\_HASH} replaces the top element with its hash,
\ttt{OP\_EQ} checks that the top two elements are equal, \ttt{OP\_CHECKSIG} verifies that the top two elements
are a valid pair of a digital signature of the transaction data and a public key hash.

The full script will be run when the output is claimed (i.e. used as input in a future transaction)
and consists of the P2PKH script, preceded by the digital signature of the transaction by its owner and a hash of
his public key. Given a digital signature \ttt{<sig>} and a public key hash \ttt{<pub>}, a transaction is valid
when the execution of the script below evaluates to \ttt{True}.
\[
  \stack{<sig> <pub> OP\_DUP OP\_HASH <pub\#> OP\_EQ OP\_CHECKSIG}
\]

To clarify, assume a scenario where Alice want to pay Bob \bitcoin ~10.
Bob provides Alice with the cryptographic hash of his public key (\ttt{<pub\#>})
and Alice can submit a transaction of \bitcoin ~10 with the following output script:
\[
  \stack{OP\_DUP OP\_HASH <pub\#> OP\_EQ OP\_CHECKSIG}
\]
After that, Bob can submit another transaction that uses this output by providing the digital signature
of the transaction \ttt{<sig>} (signed with his private key) and his public key \ttt{<pub>}.
It is easy to see that the resulting script evaluates to \ttt{True}.

\paragraph{P2SH}
A more complicated script type is \textit{pay-to-script-hash} (P2SH), where output scripts simply authenticate
against a hash of a \textit{redeemer} script \ttt{<red\#>}:
\[
  \stack{OP\_HASH <red\#> OP\_EQ}
\]

A redeemer script \ttt{<red>} resides in an input which uses the corresponding output. The following two conditions
must hold for the transaction to go through:
\begin{enumerate}
\item $\Semantics{<red>} = \stack{True}$
\item $\Semantics{<red> OP\_HASH <red\#> OP\_EQ} = \stack{True}$
\end{enumerate}
Therefore, in this case the script residing in the output are simpler, but inputs can also contain arbitrary redeemer scripts
(as long as they are of a standard ``template").

In this thesis, we will model scripts in a much more general, mathematical sense, so
we will eschew from any further investigation of properties particular to \textsc{Script}.

\subsubsection{The BitML Calculus}
Although Bitcoin is the most widely used blockchain to date, many aspects of it are poorly documented.
In general, there is a scarcity of formal models, most of which are either introductory or exploratory.

One of the most involved and mature previous work on formalizing the operation of Bitcoin
is the Bitcoin Modelling Language (BitML)~\cite{bitml}. First, an idealistic \textit{process calculus}
that models Bitcoin contracts is introduced, along with a detailed small-step reduction semantics that
models how contracts interact and its non-determinism accounts for the various outcomes.

The semantics consist of transitions between \textit{configurations}, abstracting away all the
cryptographic machinery and implementation details of Bitcoin.
Consequently, such operational semantics allow one to reason about the concurrent behaviour of
the contracts in a \textit{symbolic} setting.

The authors then provide a compiler from BitML contracts to 'standard' Bitcoin transactions, proven
correct via a correspondence between the symbolic model and the computational model operating on
the Bitcoin blockchain. We will return for a more formal treatment of BitML in Section~\ref{subsec:bitml}.

\subsubsection{Extended UTxO}
In this work, we will consider the version of the UTxO model used by IOHK's Cardano\site{www.cardano.org} blockchain.
In contrast to Bitcoin's \textit{proof-of-work} consensus protocol~\cite{bitcoin}, 
Cardano's \textit{Ouroboros} protocol~\cite{ouroboros} is \textit{proof-of-stake}.
This, however, does not concern our study of the abstract accounting model, thus we
refrain from formally modelling and comparing different consensus techniques.

The actual extension we care about is the inclusion of \textit{data scripts} in transaction
outputs, which essentially provide the validation script in the corresponding input with additional
information of an arbitrary type.

This extension of the UTxO model has already been
implemented\site{https://github.com/input-output-hk/plutus/tree/master/wallet-api/src/Ledger}, but
only informally documented\site{https://github.com/input-output-hk/plutus/blob/master/docs/extended-utxo/README.md}.
The reason to extend the UTxO model with data scripts is to bring more expressive power to UTxO-based blockchains,
hoping that it is on par with Ethereum's account-based scripting model (see Section~\ref{subsec:ethereum}).

However, there is no formal argument to support this claim, and it is the goal of this thesis
to provide the first formal investigation of the expressiveness introduced by this extension.

\subsection{Account-based: Ethereum} \label{subsec:ethereum}
On the other side of the spectrum, lies the second biggest cryptocurrency today, Ethereum~\cite{ethereum}.
In contrast to UTxO-based systems, Ethereum has a built-in notion of user addresses and operates on a
stateful accounting model. It goes even further to distinguish \textit{human accounts}
(controlled by a public-private key pair) from \textit{contract accounts} (controlled by some EVM code).

This added expressiveness is also reflected in the quasi-Turing-complete low-level stack-based bytecode language
in which contract code is written, namely the \textit{Ethereum Virtual Machine} (EVM).
EVM is mostly designed as a target, to which other high-level user-friendly languages will compile to.

\paragraph{Solidity}
The most widely adopted language that targets the EVM is \textit{Solidity},
whose high-level object-oriented design makes writing common contract use-cases (e.g. crowdfunding campaigns, auctions)
rather straightforward.

One of Solidity's most distinguishing features is the concept of a contract's \textit{gas}; a limit to the amount
of computational steps a contract can perform.
At the time of the creation of a transaction, its owner specifies a certain amount of gas the contract can consume and
pays a transaction fee proportional to it. In case of complete depletion, all global state changes are reverted.
This is a necessary ingredient for smart contract languages that provide
arbitrary looping behaviour, since non-termination of the validation phase is certainly undesirable.

If time permits, we will initially provide a formal justification of Solidity and proceed to
formally compare the extended UTxO model against it.
Since Solidity is a fully-fledged programming language with lots of features
(e.g. static typing, inheritance, libraries, user-defined types), it makes sense to 
restrict our formal study to a compact subset of Solidity that is easy to reason about.
This is the approach also taken in Featherweight Java~\cite{featherweightjava}; a subset
of Java that omits complex features such as reflection, in favour of easier behavioural reasoning
and a more formal investigation of its semantics.
In the same vein, we will try to introduce a lightweight version of Solidity, which we will refer to as
\textit{Featherweight Solidity}.

%%%%%%%%%%%%%%%%%%%%%%%%%%%%%%%%%%%%%%%%%%
\section{Methodology}
\label{sec:methodology}
%%%%%%%%%%%%%%%%%%%%%%%%%%%%%%%%%%%%%%%%%%

\subsection{Scope}
At this point, we have to stress the fact that we are not aiming for a formalization of a fully-fledged 
blockchain system with all its bells and whistles, but rather focus on the underlying accounting model.
Therefore, we will omit details concerning cryptographic operations and aspects of the actual implementation
of such a system. Instead, we will work on an abstract layer that postulates the well-behavedness of these
subcomponents, which will hopefully lend itself to more tractable reasoning and
give us a clear overview of the essence of the problem.

Restricting the scope of our attempt is also motivated from the fact that individual
components such as cryptographic protocols are orthogonal to the functionality we study here.
This lack of tight cohesion between the components of the system allows one to 
and thus can be safely factored out and formalized independently.

It is important to note that this is not always the case for every domain. A prominent example of
this are operating systems, which consist of intricately linked subcomponents (e.g. drivers, memory modules),
thus making impossible to trivially divide the overall proof into small independent ones.
In order to overcome this complexity burden, one has to invent novel ways of modular proof mechanization, as
exemplified by \textit{CertiKOS}~\cite{certikos}, a formally verified concurrent OS.

\subsection{Proof Mechanization}
Fortunately, the sub-components of the system we are examining are not no interdependent,
thus lending themselves to separate treatment.
Nonetheless, the complexity of the sub-system we care about is still high and requires rigorous investigation.
Therefore, we choose to conduct our formal study in a mechanized manner, i.e. using a proof assistant
along the way and formalizing all results in Type Theory.
Proof mechanization will allow us to discover edge cases and increase the confidence of the model under investigation.

\subsection{Agda}
As our proof development vehicle, we choose Agda~\cite{agda}, a dependently-typed total functional language
similar to Haskell~\cite{haskell}.

Agda embraces the \textit{Curry-Howard correspondence}, which states that types are isomorphic to statements in
(intuitionistic) logic and their programs correspond to the proofs of these statements~\cite{itt}.
Through its unicode-based \textit{mixfix} notational system, one can easily translate a mathematical formula
into a valid Agda type. Moreover, programs closely follow the structure of the corresponding proof, e.g. induction
in the proof manifests itself as recursion in the program.

While Agda is not ideal for large software development, its flexible notation
and elegant design is suitable for rapid prototyping of new ideas and exploratory purposes.
We estimate not to hit such a barrier of complexity,
since we will stay on a fairly abstract level which postulates cryptographic operations and other implementation details.

\paragraph{Limitation}
The main limitation of Agda lies in its lack of a proper proof automation system.
While there has been work on providing Agda with such capabilities~\cite{agdaauto},
it requires moving to a meta-programming mindset which would be an additional programming hindrance.

A reasonable alternative would be to use Coq~\cite{coq}, which provides a pragmatic
script-like language for programming \textit{tactics}, i.e. programs that work on proof contexts and can
generate new sub-goals.
This approach to proof mechanization has, however, been criticized for widening the gap between informal proofs
and programs written in a proof assistant.
This clearly goes against the aforementioned principle of \textit{proofs-as-programs}.

\subsection{The IOHK approach}
At this point, we would like to mention the specific approach taken by IOHK\site{https://iohk.io/}.
In contrast to numerous other companies currently creating cryptocurrencies, its main focus
is on provably correct protocols with a strong focus on peer-reviewing and robust implementations, rather
than fast delivery of results.
This is evidenced by the choice of programming languages (Agda/Coq/Haskell/Scala)
-- all functional programming languages with rich type systems --
and the use of \textit{property-based testing}~\cite{quickcheck} for the production code.

IOHK's distinct feature is that it advocates a more rigorous development pipeline;
ideas are initially worked on paper by pure academics,
which create fertile ground for starting formal verification in Agda/Coq for more confident results, 
which result in a prototype/reference implementation in Haskell,
which informs the production code-base (also written in Haskell) on the properties that should be tested.

Since this thesis is done in close collaboration with IOHK, it is situated on the second step of aforementioned pipeline;
while there has been work on writing out papers about the extended UTxO model, there are still no formally verified results.

\subsection{Functional Programming Principles}
One last important manifestation of the functional programming principles behind IOHK is the choice
of a UTxO-based cryptocurrency itself.

One the one hand, one can view a UTxO ledger as a dataflow diagram, whose nodes are the submitted transactions
and edges represent links between transaction inputs and outputs.
On the other hand, account-based ledgers rely on a global state and transaction have a much more complicated
specification.

The key point here is that UTxO-based transaction are just pure mathematical functions, which are much more
straightforward to model and reason about.
Coming back to the principles of functional programming, one could contrast this with the difference between
functional and imperative programs.
One can use \textit{equational reasoning} for functional programs, due to their \textit{referential transparency},
while this is not possible for imperative programs that contain side-effectful commands.

\newpage
%%%%%%%%%%%%%%%%%%%%%%%%%%%%%%%%%%%%%%%%%%
\section{Preliminary Results}
\label{sec:results}
%%%%%%%%%%%%%%%%%%%%%%%%%%%%%%%%%%%%%%%%%%

\subsection{Formal Model I: Extended UTxO} \label{subsec:eutxo}
\subsubsection{Inherently-typed validity of transactions}
\subsubsection{Scripts via Denotational Semantics}
\subsubsection{Address space as module parameter}
\subsubsection{Weakening Lemma}
\subsubsection{Example}

\subsection{Formal Model II: BitML Calculus} \label{subsec:bitml}
\subsubsection{Contracts in BitML}
\subsubsection{Small-step Semantics}
... mention paper bug in [C-Control] ...
\subsubsection{Configurations modulo permutation}
\subsubsection{Example}

\subsection{Expected Problems}
... up to permutation -> quotient types -> homotopy type theory ...

\newpage
%%%%%%%%%%%%%%%%%%%%%%%%%%%%%%%%%%%%%%%%%%
\section{Planning}
\label{sec:plan}
%%%%%%%%%%%%%%%%%%%%%%%%%%%%%%%%%%%%%%%%%%

\subsection{Extended UTxO}
... multi-currency ... 

\subsection{BitML Calculus}
... symbolic runs ... computational runs ... coherence

\subsection{UTxO-BitML Integration}

\subsection{Plutus Integration}

\subsection{Featherweight Solidity}

\subsection{Formal Comparison}
... lots of examples ...

\subsection{Proof Automation}

\subsection{Timetable}

... mention things that are really out-of-scope ...

\begin{figure*}
  \centering
  \newcommand{\months}[1]{\multicolumn{#1}{c}{\cellcolor{teal}} \\}
  \begin{tabular}{lcccccccc}
    \hline
    & 2018 & \multicolumn{7}{c}{2019} \\
    \hline
    & Dec & Jan & Feb & Mar & Apr & May & Jun & Jul \\
    \hline

    \textbf{UTxO} \\
    Basic abstract model                &      \months{2}
    Weakening lemma                     &&     \months{1}
    Extend with data scripts            &&&    \months{1}
    Multi-currency                      &&&&   \months{1}

    \textbf{BitML} \\
    Small-step Semantics                &      \months{3}
    Symbolic Model                      &&&&   \months{1}
    Computational Model                 &&&&   \months{1}
    Coherence                           &&&&&  \months{2}
    Computational Soundness             &&&&&  \months{2}

    \textbf{UTxO-BitML Integration} \\
    Compiling BitML contracts to UTxO   &&&&   \months{2}
    
    \textbf{Solidity} \\
    Featherweight Model                 &&&&   \months{1}
    Comparison framework                &&&&&  \months{2}
    
    \textbf{Evaluation} \\
    Example contracts                   &&&&&  \months{3}
    
    \textbf{Writeup} \\
    Write final thesis                  &&&&&& \months{3}
    
  \end{tabular}
  \caption{My workplan.}
  \label{fig:workplan}
\end{figure*}

\newpage
%%%%%%%%%%%%%%%%%%%%%%%%%%%%%%%%%%%%%%%%%%
%% Bibliography
%%%%%%%%%%%%%%%%%%%%%%%%%%%%%%%%%%%%%%%%%%
\nocite{*} % T0D0 remove
\bibliography{sources}

\end{document}
