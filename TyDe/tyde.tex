\documentclass[sigplan,review,screen]{acmart}\settopmatter{printfolios=true,printccs=false,printacmref=false}

%% Bibliography style
\bibliographystyle{ACM-Reference-Format}
%\citestyle{acmauthoryear}

%% Conference information
\acmConference[PL'18]{ACM SIGPLAN Conference on Programming Languages}{January 01--03, 2018}{New York, NY, USA}
\acmYear{2018}
\acmISBN{} % \acmISBN{978-x-xxxx-xxxx-x/YY/MM}
\acmDOI{} % \acmDOI{10.1145/nnnnnnn.nnnnnnn}
\startPage{1}

% Copyright
\setcopyright{none}

%%%%%%%%%%%%%%%%%%%%%%%%%%%%%%%%%%%%%%%%%%
%% Packages
%%%%%%%%%%%%%%%%%%%%%%%%%%%%%%%%%%%%%%%%%%

% URLs
\newcommand\site[1]{\footnote{\url{#1}}}

% Import Agda code
\newcommand\nat{
\begin{code}%
\>[0]\AgdaKeyword{data}\AgdaSpace{}%
\AgdaDatatype{ℕ}\AgdaSpace{}%
\AgdaSymbol{:}\AgdaSpace{}%
\AgdaPrimitiveType{Set}\AgdaSpace{}%
\AgdaKeyword{where}\<%
\\
\>[0][@{}l@{\AgdaIndent{0}}]%
\>[2]\AgdaInductiveConstructor{zero}%
\>[8]\AgdaSymbol{:}\AgdaSpace{}%
\AgdaDatatype{ℕ}\<%
\\
%
\>[2]\AgdaInductiveConstructor{suc}%
\>[8]\AgdaSymbol{:}\AgdaSpace{}%
\AgdaDatatype{ℕ}\AgdaSpace{}%
\AgdaSymbol{→}\AgdaSpace{}%
\AgdaDatatype{ℕ}\<%
\\
%
\\[\AgdaEmptyExtraSkip]%
\>[0]\AgdaOperator{\AgdaFunction{\AgdaUnderscore{}+\AgdaUnderscore{}}}\AgdaSpace{}%
\AgdaSymbol{:}\AgdaSpace{}%
\AgdaDatatype{ℕ}\AgdaSpace{}%
\AgdaSymbol{→}\AgdaSpace{}%
\AgdaDatatype{ℕ}\AgdaSpace{}%
\AgdaSymbol{→}\AgdaSpace{}%
\AgdaDatatype{ℕ}\<%
\\
\>[0]\AgdaInductiveConstructor{zero}%
\>[7]\AgdaOperator{\AgdaFunction{+}}\AgdaSpace{}%
\AgdaBound{n}\AgdaSpace{}%
\AgdaSymbol{=}\AgdaSpace{}%
\AgdaBound{n}\<%
\\
\>[0]\AgdaInductiveConstructor{suc}\AgdaSpace{}%
\AgdaBound{m}%
\>[7]\AgdaOperator{\AgdaFunction{+}}\AgdaSpace{}%
\AgdaBound{n}\AgdaSpace{}%
\AgdaSymbol{=}\AgdaSpace{}%
\AgdaInductiveConstructor{suc}\AgdaSpace{}%
\AgdaSymbol{(}\AgdaBound{m}\AgdaSpace{}%
\AgdaOperator{\AgdaFunction{+}}\AgdaSpace{}%
\AgdaBound{n}\AgdaSymbol{)}\<%
\end{code}
}


\begin{document}
\sloppy % for proper justification (no overflows to the right)

%%%%%%%%%%%%%%%%%%%%%%%%%%%%%%%%%%%%%%%%%%
%% Title
%%%%%%%%%%%%%%%%%%%%%%%%%%%%%%%%%%%%%%%%%%
\title{Formal Investigation of the Extended UTxO Model (Extended Abstract)}

%%%%%%%%%%%%%%%%%%%%%%%%%%%%%%%%%%%%%%%%%%
%% Authors
%%%%%%%%%%%%%%%%%%%%%%%%%%%%%%%%%%%%%%%%%%
\author{Orestis Melkonian}
\orcid{0000-0003-2182-2698}
\affiliation{
  \department{Information and Computing Sciences}
  \institution{Utrecht University}
}
\email{melkon.or@gmail.com}

\author{Wouter Swierstra}
%\orcid{????-????-????-????}
\affiliation{
  \department{Information and Computing Sciences}
  \institution{Utrecht University}
}
\email{w.s.swierstra@uu.nl}

\author{Manuel M.T. Chakravarty}
%\orcid{????-????-????-????}
\affiliation{
  \institution{Input Output HK}
}
\email{manuel.chakravarty@iohk.io}

%%%%%%%%%%%%%%%%%%%%%%%%%%%%%%%%%%%%%%%%%%
%% Abstract
%%%%%%%%%%%%%%%%%%%%%%%%%%%%%%%%%%%%%%%%%%
%\begin{abstract}
%\end{abstract}

%% 2012 ACM Computing Classification System (CSS) concepts
%% Generate at 'http://dl.acm.org/ccs/ccs.cfm'.
\begin{CCSXML}
<ccs2012>
<concept>
<concept_id>10002978.10002986</concept_id>
<concept_desc>Security and privacy~Formal methods and theory of security</concept_desc>
<concept_significance>500</concept_significance>
</concept>
<concept>
<concept_id>10002978.10002986.10002990</concept_id>
<concept_desc>Security and privacy~Logic and verification</concept_desc>
<concept_significance>500</concept_significance>
</concept>
<concept>
<concept_id>10003752.10010124.10010131</concept_id>
<concept_desc>Theory of computation~Program semantics</concept_desc>
<concept_significance>300</concept_significance>
</concept>
</ccs2012>
\end{CCSXML}

\ccsdesc[500]{Theory of computation~Program semantics}
\ccsdesc[500]{Security and privacy~Formal methods and theory of security}
\ccsdesc[500]{Security and privacy~Logic and verification}
%% End of generated code

%% Keywords (comma separated list)
%\keywords{}

\maketitle

%%%%%%%%%%%%%%%%%%%%%%%%%%%%%%%%%%%%%%%%%%
\section{Introduction}
\label{sec:intro}
%%%%%%%%%%%%%%%%%%%%%%%%%%%%%%%%%%%%%%%%%%
Blockchain technology has seen a plethora of applications during the past few years~\cite{mpc,zeroproof,fair},
but has also unveiled a new source of vulnerabilities\site{
https://en.wikipedia.org/wiki/The_DAO_(organization)
} arising from the distributed execution of \textit{smart contracts} (programs that run on the blockchain).
Since many of these applications deal with transactions of significant funds, it is crucial
that we can formally reason about their (concurrent) behaviour.

This opens up a lot of opportunities for novel applications of formal verification techniques.
In particular, we believe that a language-based, type-driven approach to contract development
constitutes an effective way to make their execution more predictable.
To this end, we attempt to lay the foundations for a mechanized formal framework,
where one can verify that certain undesirable scenarios are impossible.

We formulate an accounting model for ledgers based on \textit{unspent transaction outputs} (UTxO),
the ledger model underlying Bitcoin~\cite{bitcoin} and many other blockchains.
We conduct our study in Agda~\cite{agda},
exploiting its expressive dependent type system to mechanically enforce desired properties statically.
An executable specification of our formal development is available on Github\site{
https://github.com/omelkonian/formal-utxo/
}.

%%%%%%%%%%%%%%%%%%%%%%%%%%%%%%%%%%%%%%%%%%
\section{Formal Model}
\label{sec:model}
%%%%%%%%%%%%%%%%%%%%%%%%%%%%%%%%%%%%%%%%%%
Our formalization closely follows the abstract accounting model for UTxO-based cryptocurrencies presented in \cite{utxo},
which leaves out details of other technical components of the blockchain such as cryptographic operations.
We further extend the original formulation to cover the extensions employed by the Cardano blockchain platform~\cite{eutxo}.
Cardano extends Bitcoin's UTxO model with data scripts on transaction outputs,
in an effort to bring it on par with Ethereum's expressive account-based scripting model~\cite{ethereum},
as well as support for multiple cryptocurrencies on the same ledger~\cite{multicurrency}.

\paragraph{Transactions \& Ledgers.}
For simplicity, we model monetary quantities and hashes as natural numbers.
We treat the type of addresses as an abstract module parameter equipped with an injective hash function.
Transactions consist of a list of outputs, transferring a monetary value to an address, and a list of inputs
referring to previous outputs:
\basic{}

Both inputs and outputs carry authorization scripts; for a transaction to consume an unspent output, the
result of the validator script has to evaluate to \inlineTrue{}, given the current state of the ledger and
additional information provided by the redeemer and data scripts:
\auth{}

A ledger consists of a list of transactions, whose \textit{unspent transaction outputs} we can recursively compute:
\utxo{}

\paragraph{Validity.}
We are now ready to encode the validity of a transaction with respect to a given ledger as a dependent data type.
For the sake of brevity, we present only two such conditions, namely that inputs refer to existing unspent outputs
and all authorizations succeed:
\valid{}
Other validity conditions include that no output is spent twice (\inlineNDS{})
and transactions preserve total values (\inlinePV{}).

It is now possible to characterize a well-formed \inlineLedger{},
by requiring a validity proof along with each insertion to the list of transactions.
Exposing only this type-safe interface to the user will ensure one can only construct valid ledgers.

%%%%%%%%%%%%%%%%%%%%%%%%%%%%%%%%%%%%%%%%%%
\section{Meta-theory}
\label{sec:meta}
%%%%%%%%%%%%%%%%%%%%%%%%%%%%%%%%%%%%%%%%%%
Apart from being able to define correct-by-construction ledgers, we can prove further meta-theoretical results over our
existing formulation.

\paragraph{Weakening.}
Given a suitable injection on addresses, we prove a weakening lemma, stating that a valid ledger
parametrized over some addresses will remain valid even if more addresses become available:
\weakening{}
Weakening consists of traversing the ledger's outputs and transporting all addresses via the supplied injection;
in order to keep references intact, the injection has to also preserve the original hashes\footnote{
A practical case of such weakening is migrating from a 32-bit word address space to a 64-bit one.
}.

\paragraph{Combining.}
Ideally, one would wish for a modular reasoning process, where it is possible to examine subsets of 
unrelated transactions in a compositional manner.

We provide a ledger combinator that interleaves two \textit{separate} ledgers.
Due to lack of space, we eschew from giving the formal definition of the separation connective \inlineAst{}.
Briefly, two ledgers are separate if they do not share any common transaction and the produced interleaving 
does not break previous validator scripts (since they will now execute on a different ledger state).
These conditions are necessary to transfer the validity of the two sub-ledgers to a proof of validity of the merged ledger.
\combining{}
The notion of weakening we previously defined proves rather useful here,
as it allows merging two ledgers acting on different addresses.

%%%%%%%%%%%%%%%%%%%%%%%%%%%%%%%%%%%%%%%%%%
\section{Discussion}
\label{sec:discussion}
%%%%%%%%%%%%%%%%%%%%%%%%%%%%%%%%%%%%%%%%%%
\paragraph{Proof Automation.}
Although we have made it possible to express desired ledger properties in the type system,
users still need to manually discharge tedious proof obligations.
In order to make the proof process more ergonomic, we can prove that the involved propositions are \textit{decidable},
thus defining a decision procedure for closed formulas that do not contain any free variables~\cite{proofbyreflection};
we have already proven decidability of the validity conditions\footnote{
There is an example construction of a valid ledger in the code repository,
where our decision procedure automatically discharges all required proofs.
}
and wish to also cover the propositions appearing in weakening and combining.

\paragraph{Comparison with Ethereum.}
It would be interesting to conduct a more formal comparison between UTxO-based and account-based ledgers,
relying on previous work on \textit{chimeric ledgers}~\cite{chimeric} that gives a translation between these two approaches.
Note that implementing this translation on our inherently-typed representation would
guarantee that we only produce \textbf{valid} UTxO ledgers.

\paragraph{Towards verification of smart contracts.}
Although our framework gives a formal foundation for UTxO-based ledgers,
reasoning about the high-level behaviour of smart contracts is still out of reach.
The quest for a mathematical model that captures the subtleties of contract behaviour
and is amendable to mechanized verification is still an open problem,
but there seems to be a consensus that formal methods lead to the most promising direction~\cite{smartcontracts}.

\textsc{Scilla}, an intermediate-level language for smart contracts,
has a formal semantics based on communicating automata that has proven adequate
to mechanically verify \textit{safety} and \textit{liveness} properties~\cite{scilla}.

The Bitcoin Modelling Language (BitML), an idealistic process calculus for Bitcoin contracts,
is accompanied by a small-step reduction semantics
and a symbolic model of participant strategies that is intuitive to work with~\cite{bitml}.
The authors also provide a compiler from high-level BitML contracts to low-level Bitcoin transactions,
along with a compilation correctness theorem:
\textit{computational attacks on compiled contracts are also observable in the symbolic model}.
We are, in fact, currently formalizing the BitML calculus and its symbolic model in Agda\site{
https://github.com/omelkonian/formal-bitml
} and plan to mechanize compilation down to our formal UTxO model instead.

%%%%%%%%%%%%%%%%%%%%%%%%%%%%%%%%%%%%%%%%%%
%% Acknowledgements
%%%%%%%%%%%%%%%%%%%%%%%%%%%%%%%%%%%%%%%%%%
\begin{acks}
We would like to thank Philip Wadler and Michael Peyton Jones for helpful discussion and IOHK for financial support.
\end{acks}

%%%%%%%%%%%%%%%%%%%%%%%%%%%%%%%%%%%%%%%%%%
%% Bibliography
%%%%%%%%%%%%%%%%%%%%%%%%%%%%%%%%%%%%%%%%%%
\bibliography{sources}

\end{document}
